
%! TEX root = paper.tex

\section{INTRODUCTION}

Micro Aerial Vehicles (MAVs) are about to play a major role in tasks like search and rescue, environment monitoring, security surveillance, inspection and goods delivery (Amazon).  However, for such operations, navigating based on GPS information only is not sufficient. Fully autonomous operation in cities or other dense environments requires MAVs to fly at low altitudes where GPS signals are often shadowed or indoors, and to actively explore unknown environments while avoiding collisions and creating maps. Precisely autonomous operations requires MAVs to rely on alternative localization systems. For minimal weight, power consumption and budget a single camera can be used for this propose.\\

Real-time monocular Visual Odometry (VO) algorithms can be used to estimate the 6 DoF pose of a camera relative to its surroundings. This is attractive for many applications such as mobile robotics (and not only aerial) and Augmented Reality (AR) because cameras are small and self-contained and therefore easy to attach to autonomous robots or AR displays. Further, they are cheap, and are now often pre-integrated into mobile computing devices such as PDAs, phones and laptops.\\

SVO (Semi-direct Visual Odometry) \cite{Forster2014} is a very fast VO algorithm able to run at more than 300 frames per second on a consumer laptop. It builds a map based on keyframes and salient points. Most monocular VO are feature-based where scale and rotation invariant descriptors (SIFT, SURF\ldots) are extracted and matched in order to recover the motion from frame to frame while finally refining the pose with reprojection error minimization with the map. SVO uses a different approach by using direct methods. Instead of matching descriptors, it uses intensity gradients to minimize the error between patches around detected salient points to estimate the frame to frame transformation. Finally, it uses Bundle Adjustment to align with the map and avoid or minimize drift.\\

The main problem with most existing monocular VO implementations (including SVO) is a lack of robustness. Rapid camera motions, occlusion, and motion blur (phenomena which are common in all but the most constrained experimental settings) can often cause the tracking to fail. While this is inconvenient with any tracking system, tracking failure is particularly problematic for VO systems: not only is the camera pose lost, but the estimated map could become corrupted as well. \\

This problem is accentuated during a fast  agile maneuver (e.g., a flip) and so a good relocalization is important when these are intended to be performed. \\
