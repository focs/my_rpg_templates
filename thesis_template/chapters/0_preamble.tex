%---------------------------------------------------------------------------
% Table of contents

 \setcounter{tocdepth}{2}
 \tableofcontents
 \cleardoublepage

%---------------------------------------------------------------------------
% List of Figures

 % \addcontentsline{toc}{chapter}{List of Figures}
 % \listoffigures
 % \clearpage

%---------------------------------------------------------------------------
% List of Tables

 % \addcontentsline{toc}{chapter}{List of Algorithms}
 % \listofalgorithms
 % \clearpage

%---------------------------------------------------------------------------
% Abstract

\chapter*{Abstract}
 \addcontentsline{toc}{chapter}{Abstract}

 Visual odometry techniques  are largely used on robotic vehicle and more specifically on MAVs (Micro Aerial Veicle). While being a very interesting solution this algorithm are known to get lost when track is lost due to blurred images or large frame to frame changes becoming the kidnapped robot problem. In the following work we are going to study and propose some relocalization solutions for Visual Odometry algorithms. This work is intended to work with SVO (Semi-direct Visual Odometry) but the proposed framework can be used with other methods. The relocalization method from PTAM is used as a base line and new alternatives based on space geometry and machine lairing are proposed.

 \cleardoublepage

%---------------------------------------------------------------------------
% Symbols

%\chapter*{Nomenclature}\label{chap:symbole}
%\addcontentsline{toc}{chapter}{Nomenclature}
%
%\section*{Notation}
%  \begin{tabbing}
%    \hspace*{1.6cm}   \= \kill
%    $\mathbf{J}$       \> Jacobian \\[0.5ex]
%    $\mathbf{H}$       \> Hessian \\[0.5ex]
%    $\mathbf{T}_{WB}$  \> coordinate transformation from $B$ to $W$ \\[0.5ex]
%    $\mathbf{R}_{WB}$  \> orientation of $B$ with respect to $W$ \\[0.5ex]
%    $_W\mathbf{t}_{WB}$\> translation of $B$ with respect to $W$, expressed in coordinate system $W$ \\[0.5ex]
%  \end{tabbing}
%  
%Scalars are written in lower case letters ($a$), vectors in lower case bold letters ($\mathbf{a}$) and matrices in upper case bold letters ($\mathbf{A}$).
%
%\section*{Acronyms and Abbreviations}
%  \begin{tabbing}
%    \hspace*{1.6cm}  \= \kill
%    RPG     \> Robotics and Perception Group \\[0.5ex]
%    DoF     \> Degree of Freedom \\[0.5ex]
%    IMU     \> Inertial Measurement Unit \\[0.5ex]
%    MAV     \> Micro Aerial Vehicle \\[0.5ex]
%    ROS     \> Robot Operating System \\[0.5ex]
%  \end{tabbing}
%
%\clearpage

%---------------------------------------------------------------------------
